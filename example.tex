\documentclass[submission,copyright,creativecommons]{eptcs}
\providecommand{\event}{COMP4630 ASE} % Name of the event you are submitting to
\usepackage{breakurl} % Not needed if you use pdflatex only.
\usepackage{todonotes}
\usepackage{amsmath,amssymb}
\usepackage{bussproofs}
\usepackage{newtxtext}
\usepackage{newtxmath}
\usepackage{enumitem}
\usepackage{url}
\usepackage{amsmath}

\usepackage[fleqn]{mathtools}

%Semantic Tree
\usepackage{tikz}
\usetikzlibrary{positioning,arrows,calc}
\tikzset{modal/.style={>=stealth?,shorten >=1pt,shorten <=1pt,auto,node distance=1.5cm, semithick}, world/.style={circle,draw,minimum size=0.5cm,fill=gray!15}, point/.style={circle,draw,inner sep=0.5mm,fill=black}, reflexive above/.style={->,loop,looseness=7,in=120,out=60}, reflexive below/.style={->,loop,looseness=7,in=240,out=300}, reflexive left/.style={->,loop,looseness=7,in=150,out=210}, reflexive right/.style={->,loop,looseness=7,in=30,out=330}}

\newtheorem{lemma}{Lemma}
\title{Global Deducibility and Local Deducibility in Modal logic in HOL4}
\author{Zhuo Chen
\institute{
University of Australian National University\\
Canberra, Australia}
\email{zhuo.chen1@anu.edu.au}
}
\def\titlerunning{Global Deducibility and Local Deducibility in Modal logic in HOL4}
\def\authorrunning{Zhuo Chen}
\begin{document}
\maketitle

\begin{abstract}
In modal logic, there are two different deducibility notations: global deducibility 
and local deductibility. In HOL4, local deducibility has been defined along with proofs 
of some of its properties. In this report, we present how we defined global deducibility
and proved that given no assumptions other than axioms, we have that global deducibility and local deducibility
are equivalent to each other. This can be used as a tool to assist in proving the completeness 
of modal logic. 
\end{abstract}

\section{Introduction}

Many existing textbook completeness proofs of modal logic naively say if $\models (\wedge \Gamma) \rightarrow \varphi$ 
then $\emptyset \vdash (\wedge \Gamma) \rightarrow \varphi$. 
This is not right because they are confusing local deducibility with 
global deducibility. The correct statement should be if
$\models (\wedge \Gamma) \rightarrow \varphi$ 
then $\emptyset \vdash_l (\wedge \Gamma) \rightarrow \varphi$,
where $\vdash_l$ means locally deducible. 
We give global deducibility and local deducibility different definitions 
to distinguish them.
We define global deducibility ($\vdash$) to be $\Gamma \vdash \varphi$ if $\Gamma$ is true everywhere
, then $\varphi$ is true everywhere.
We define local deducibility ($\vdash_l$) to be $\Gamma \vdash_l \varphi$ 
if $\Gamma$ is true at one world then $\varphi$ is also true at that world.

We say that $\Gamma \vdash_l \varphi$ if and only if there exists a 
finite subset $\{\psi_1, \psi_2,\dots, \psi_n\} \subset \Gamma$ 
such that $\emptyset \vdash (\psi_1 \wedge \psi_2 \wedge \dots \wedge \psi_n) \rightarrow \varphi$.


Using the transformation between local and global deducibility we 
are still able to do similar completeness proof but in a more rigid way. However, not everyone is convinced.

We want to formalise our proof to show that our proof is correct and 
can be trusted. The first step is to show the relation between local and global deducibility,
which is what we did in this report. 
In this report, we also discussed our failed attempt, which provides 
valuable thinking procedure. 


\section{Background}
The motivation of our work is to enable us to do the 
completeness proof for modal logic using local and global deducibility models. 
Many existing textbook proofs for completeness of modal logic mix 
local deducibility and global deducibility up as one thing and prove the completeness 
through this easy way out. We want to show that they should not 
think that local deducibility and global deducibility are the same by formalising 
the proof. 

Our work is built on top of some existing work on modal logic in HOL4\cite{yiming} by Yiming Xu.
In the following sections, we will introduce in Yiming's work, how logic connectives 
are defined, how the instantiation of axioms is implemented and how local 
deducibility is defined. 

\subsection{Modal Logic in HOL4}
\subsubsection{Definition}
\label{mldef}
\begin{verbatim}
 val _ = Datatype`
 form
 = VAR num
 | DISJ form form
 | FALSE
 | NOT form
 | DIAM form`;
\end{verbatim}
 
We use $VAR$ to notate atoms which only 
constitute one value. 
$\vee, \bot, \neg$ and $\diamond$ are taken as primitive connectives.
All the other connectives are defined in terms of these four connectives.
$\rightarrow, \leftrightarrow, \wedge$ and $\top$ are defined in the same way as 
in propositional logic. The important thing to notice is that we define 
$\square p$ to be $\neg \diamond \neg p$. 

All the definitions here are on the syntactical level. 


\subsubsection{Instantiation: subst}
The instantiation of axioms are implemented by a function: \texttt{subst}.
\texttt{subst}'s definition in HOL4 is shown below:
\begin{verbatim}
 val subst_def =
 Define
 `subst f FALSE = FALSE /\
 subst f (VAR p) = f p /\
 subst f (DISJ form1 form2) = DISJ (subst f form1) (subst f form2) /\
 subst f (NOT form) = NOT (subst f form) /\
 subst f (DIAM form) = DIAM (subst f form)`;
\end{verbatim}

Each line of the definition represents one rule for \texttt{subst}:
\begin{itemize}
\item \texttt{subst} does not change falsum 
\item \texttt{subst} applies the swap function to the variable
\item \texttt{subst} performs the same substitution over the two disjunctions 
if the main connective is $\vee$
\item \texttt{subst} lifts $\neg$ and perform the same substitution 
to the inner formulae
\item \texttt{subst} lifts $\diamond$ and perform the same substitution 
to the inner formulae
\end{itemize}

\subsubsection{KGproof}
KGproof represents local deducibility ($\vdash_l)$.
\paragraph{Notation} We notate a list of formulae (p) locally deducible 
from a set of axioms (Ax) by $Ax \vdash_l p$\footnote{We are notating in this way because that follows 
from the code, in logic proving we usually write axioms in the definition rather than the assumption. The same goes 
for all the notations in this report}. 
\paragraph{Rules} We have emtpy p ($Ax \vdash_l []$) as the base case. 
Appending new elements onto the list if they can be obtained from the existing 
elements in the list by the following derivation rules: modus ponens, Ax (instantiation), 
necessitation, K axiom (instantiation), Dual axiom (instantiation) (two directions are written separately as 
two rules), propositional tautology and element in the list instantiation. 
Note that the last rule holds because all the elements in the list only 
come from axioms (including Ax, K and Dual axioms) or propositional tautologies, 
of whose all instances are true, to begin with. 



\section{Construction of Global Deducibility}
We first defined global deducibility (gtt, $\vdash$) in our system.
Note that all the logic connectives are the same as the ones in section~\ref{mldef}
\subsection{Notation} 
We notate a formula (f) globally deducible from a set of axioms (Ax)
and a set of assumptions ($\Gamma$) by $Ax, \Gamma \vdash f$. 
\subsection{Rules}
We then add rules for gtt. 
\begin{itemize}
 \item (Id) If f is in $\Gamma$ then $Ax, \Gamma \vdash f$. 
 \item (Ax) All instances of $Ax$ are gtt provable. $Ax, \Gamma \vdash axf$ where $axf$ is 
 any arbitrary instance of $Ax$. Instantiation is 
 \item (MP) If $Ax, \Gamma \vdash f1$ and $Ax, \Gamma \vdash (f1 \rightarrow f2)$, then 
 we have $Ax, \Gamma \vdash f2$
 \item (Nec) If $Ax, \Gamma \vdash f$ then $Ax, \Gamma \vdash \square f$ 
\end{itemize}

\section{Before Proof: key ideas}
We want to prove that KGproof and gtt \footnote{with empty assumption $\Gamma$, 
we will call this gtt as gttE to simplify the further writing in this report}
 are as effective 
to show the equivalence between global deducibility (with no assumptions $\Gamma$) and local deductibility. 

To do so, we need to prove that gttE
can prove everything that KGproof can prove; 
we also need to prove that KGproof can prove everything that gttE can prove.

The proofs will have a top-level induction on all the rules of gtt or KGproof depending on which 
proof we are doing. We then use the deduction rules of the other one and some helper lemmas 
to prove that the rules we get from the top level induction.

\section{$\vdash$ to $\vdash_l$: first attempt}
We have one failed attempt for the proof, which is crucial as we 
realised the problem with this attempt and fixed it, thus obtained the final proof 
in section~\ref{final}.


\subsection{Proof Goal}

Our attempt was trying to prove that gttE is at least as strong as KGproof.
In other words, we want to prove that everything that can be proved by KGproof with axiom set $Ax$
must be provable using gttE with the same axiom set $Ax$ with a few additional axioms.
Note that we need the additional axioms here because KGproof encoded some axioms (such as K axiom) 
in their rules, rather than taking them from external. If we want to achieve the same 
thing we have to take these axioms in by ourselves too. 

We then formalise our idea using HOL4 and wrote the following goal\footnote{The exclamation mark (!) here means for all here} to prove:


$$Ax \vdash_l p \Rightarrow \forall f. f \in p \Rightarrow (Ax\cup KAxioms) \vdash f$$

Note that the KAxioms here is referring to the set of all the classical propositional 
logic axioms (propositional tautologies) and K axioms. 

\subsection{Proof Difficulty}
\label{difficulty}
We managed to prove all the KGproof rules
\footnote{the proofs are similar to the final proof so we will skip them here} 
except $Ax \cup KAxioms \vdash \diamond p \rightarrow \neg \square \neg p$\footnote{every p mentioned 
in this way is of any arbitrary value} and 
$Ax \cup KAxioms \vdash \neg \square \neg p \rightarrow \diamond p$. 


\subsection{Lemmas to Help} 
We try to write lemmas to overcome the difficulty mentioned above.
We first observe that using gttE, proving $Ax \cup KAxioms \vdash \diamond p \rightarrow \neg \square \neg p$ and 
$Ax \cup KAxioms \vdash \neg \square \neg p \rightarrow \diamond p$ is equivalent to 
proving $Ax \cup KAxioms \vdash \diamond p \leftrightarrow \neg \square \neg p$ then split them.

We then expand the syntactical definition of $\square$ 
\footnote{mentioned in Background, Modal Logic in HOL4, Definition: $\square p = \neg \diamond \neg p $} and 
get new goal $Ax \cup KAxioms \vdash \diamond p \leftrightarrow \neg \neg \diamond \neg \neg p$ 

The double negation on the outer side is easy to remove using the lemma 
\texttt{gtt\_double\_imp\_self}, which essentially says $KAxioms \vdash (p \leftrightarrow \neg \neg p)$
for any formulae $p$. The mismatch of axiom sets of the lemma and our goal is easy 
to solve by the extension lemma~\ref{ext}. The 
extension lemma shows that extending axiom set preserves the gtt provable terms. So what we need to show now\footnote{MP is 
used implicitly here to simplify our goal} is 
$Ax \cup KAxioms \vdash \diamond p \leftrightarrow \diamond \neg \neg p$ 

In order to prove that, we tried to prove two
\footnote{Other proved but unused lemmas are in the "Unused Lemmas" section in the attached file "glbCompletenssScript.sml"} seperate lemmas: \texttt{gtt\_add\_DIAM}
and \texttt{gtt\_subst\_eqv\_terms}.

\paragraph{gtt\_add\_DIAM} If $Ax \vdash p1 \rightarrow p2$ then $Ax \vdash \diamond p1 \rightarrow \diamond p2$.
This lemma tries to attach diamonds inside an implication. 
We run into issues immediately because of the first subgoal of this 
proof, which is to prove that if from our axioms we have one instance that 
is $p1 \rightarrow p2$ (which means $Ax \vdash p1 \rightarrow p2$ by 
deduction rules of gtt), then we can have $Ax \vdash \diamond p1 \rightarrow \diamond p2$.
This subgoal is the same as our lemma! So this lemma is not provable. 

\paragraph{gtt\_subst\_eqv\_terms} If $KAxioms \vdash p1 \leftrightarrow p2$ 
then $KAxioms \vdash p3[v:=p1] \leftrightarrow p3[v:=p2]$. This 
lemma says if we substitute two formulae which implies each other seperately 
into a same formula, then the two new formulae obtained after substitution 
should imply each other too. We proved a lemma called \texttt{gtt\_double\_imp\_self} 
which proves $KAxioms \vdash p3 \leftrightarrow p3$, to show that 
every formulae implies themselves in both directions. We then start the 
proof by inducting on the structure of $p3$, of whose top level 
connective can either be nothing (it's an atom with constructor \texttt{VAR}
or $\bot$), $\vee, \neg$ or $\diamond$. We then prove each subgoal:
\begin{itemize}
 \item \texttt{VAR}: Proved by applying the definition of substitution.
 \item $\vee$: Proved by using the propositional tautology which says 
 $((x1 \leftrightarrow x2) \rightarrow (y1 \leftrightarrow y2)) \rightarrow ((x1 \vee y1)\leftrightarrow(x2 \vee y2)) $ (proved in lemma \texttt{ptaut\_disj\_double\_imp})
 So we have that if $p3$ is in the form $x \vee y$ then we have 
 $p3[v:=p1] \leftrightarrow p3[v:=p2]$ where $p3[v:=p1] = x1 \vee y1$ and $p3[v:=p2] = x2 \vee y2$.
 \item $\bot$: Proved by using the propositional tautology that $\bot \leftrightarrow \bot$.
 \item $\neg$: Proved by using the propositional tautology that $(a \leftrightarrow b)
 \rightarrow (\neg a \leftrightarrow \neg b)$.
 \item $\diamond$: we get stuck because this is in the same structure as the previous lemma
 (gtt\_add\_DIAM), which is discussed and shown as unprovable. The detailed condition of this 
 subgoal in HOL4 is shown below:
 \begin{verbatim}
 0. !p1 p2.
 gtt KAxioms EMPTYSET (DOUBLE_IMP p1 p2) ==>
 gtt KAxioms EMPTYSET (DOUBLE_IMP (subst (\v. p1) p3) (subst (\v. p2) p3)) (*)
 1. gtt KAxioms EMPTYSET (DOUBLE_IMP p1 p2)
 ------------------------------------
 gtt KAxioms EMPTYSET
 (DOUBLE_IMP (DIAM (subst (λv. p1) p3)) (DIAM (subst (λv. p2) p3)))
 
 \end{verbatim}

 Using our notation in this paper, the above code is equivalent to saying we have the followings:

 Assumption 0: $\forall p1 p2. 
 ((KAxioms \vdash p1 \leftrightarrow p2) \Rightarrow (KAxioms \vdash p3[v:=p1]\leftrightarrow p3[v:=p2]))$;

 Assumption 1: $KAxioms \vdash p1 \leftrightarrow p2$

 Goal: $KAxioms \vdash \diamond p3[v:=p1] \leftrightarrow \diamond p3[v:=p2]$


 Another way of thinking 
 about this problem is that we can try to use necessitation rule to push $\square$
 into assumption 1 and then use assumption 0 to obtain the boxed version of 
 line (*). Now seems like we just need to expand the definition of $\square$ then convert that 
 back to $\diamond$. After expanding, we realised that what we need to show is that 
 we can get from $\neg \diamond \neg p$ to $\diamond p$. This is what we want to prove 
 using this lemma at the beginning!(Section~\ref{difficulty}).
\end{itemize}

We end up in a loop, which suggests that this lemma is not able to help prove the desired 
theorem discussed in section~\ref{difficulty}.

\subsection{Analysis}

We then did some pen and paper analysis about this phenomenon.

Ideally, we should be able to obtain a $\diamond$ version of K axiom,
which is what our first lemma \texttt{gtt\_add\_DIAM} trying to do.

So we want to start from $\vdash \varphi \rightarrow \psi$ and 
get $\vdash \diamond \varphi \rightarrow \diamond \psi$. 
Our thinking procedure is shown below:

\begin{prooftree}
 \AxiomC{$\vdash \varphi \rightarrow \psi$} 
 \UnaryInfC{$\vdash \neg \psi \rightarrow \neg \varphi$}
 \UnaryInfC{$\vdash \square (\neg \psi \rightarrow \neg \varphi)$}
 \UnaryInfC{$\vdash \square \neg \psi \rightarrow \square \neg \varphi$}
 \UnaryInfC{$\vdash \neg \square \neg \varphi \rightarrow \neg \square \neg \psi$}
 \UnaryInfC{$\vdash \diamond \varphi \rightarrow \diamond \psi$}
\end{prooftree}

However, we can not perform such deduction in our system because we have no 
rule to convert $\neg \square \neg p$ to $\diamond p$ in the last step. 
This step is exactly the lemma that we are not able to prove.

This indicates that we must have Dual axiom. Our global deducibility 
is not actually complete without the Dual axiom here because we defined 
$\diamond$ to be primitive and have no way to convert $\neg \square \neg p$ to $\diamond p$.
If we have $\square$ as primitive, then we can expand the definition of $\diamond$ and prove 
this easily using K axiom of $\square$.



\section{Final Proof}
\label{final}
For the final proof, we add the Dual axiom to the attempted proof and proved 
the desired equivalence. 

\subsection{New Proof Goal}
The new proof goal is mostly the same as the previous attempt, but with an additional 
axiom passed to gttE - the Dual axiom.

We then have a new formalised goal as shown below, where KDAxioms is the union 
of KAxioms and Dual axiom.
$$Ax \vdash_l p \Rightarrow \forall f. f \in p \Rightarrow (Ax\cup KDAxioms) \vdash f$$


\subsection{Lemmas}
To assist the final equivalence proof, we first proved a few lemmas.

\begin{lemma}[gtt\_double\_imp\_decompose\_ltr]
(KDAxioms $\subseteq$ Ax) $\Rightarrow$ Ax, G $\vdash$ A $\leftrightarrow$ B $\Rightarrow$ Ax, G $\vdash$ (A $\rightarrow$ B)
\label{ltr}
\end{lemma}

\begin{lemma}[gtt\_double\_imp\_decompose\_rtl]
 (KDAxioms $\subseteq$ Ax) $\Rightarrow$ Ax, G $\vdash$ A $\leftrightarrow$ B $\Rightarrow$ Ax, G $\vdash$ (B $\rightarrow$ A)
 \label{rtl}
 \end{lemma}

\begin{lemma}[subst\_ptaut]
 If p is an instance of a propositional tautology, then Ax, KDAxioms, G $\vdash$ p 
\label{sp}
\end{lemma}

\begin{lemma}[subst\_self]
 Applying identity substitution function to any term does not change that term.
 In other words, every term is an instance of themself. 
\label{ss}
\end{lemma}

\begin{lemma}[gtt\_Ext]
 Ax, G $\vdash$ f $\Rightarrow$ Ax$\cup$Ext, G $\vdash$ f 
\label{ext}
\end{lemma}

\subsection{gTk: Global Deducibility Simulates Local Deducibility}
\textbf{Proof Goal} $Ax \vdash_l p \Rightarrow Ax \cup KDAxioms \vdash f$\footnote{More 
logical way is to remove all the 
axioms on the left-hand side, they are only listed here because that's how gtt and KGproof 
are defined in HOL4}. Note that gtt here is not making any assumptions.


We prove this theorem by inducting on KGproof rules on the top level and 
then prove that gttE can prove each rule as a subgoal.
\begin{itemize}
 \item (Nec) We have the same modus ponens rule in gttE (Nec) which can prove it directly.
 \item (Ax: with instantiation) We have the same rule in gttE (Ax) which can prove it directly.
 
 \item (MP) We have the same modus ponens rule in gttE (MP) which can prove it directly.

 \item (K Axiom) Proved by instantiating K axiom then using a gttE rule (Ax). 
 
 \item (Dual Axiom) Proved by instantiating Dual axiom then using a gttE rule (Ax). 
 There is an intermediate step to get one direction implication from the 
 double implication because in KGproof the two directions of implications of Dual axiom 
 are coded as two separate rules. The intermediate step used Lemma~\ref{ltr} and Lemma~\ref{rtl}
 
 \item (Ax: propositional tautology) Proved by 1) using Lemma~\ref{sp} to show that all 
 instances of propositional tautologies are gtt provable; 2) then using Lemma~\ref{ss}
 to show that each propositional tautology definition is its instance; 3) lastly using 
 Lemma~\ref{ext} and HOL4 theorem about the commutativity of set unions to 
 get the final proof.

 \item (Ax: without instantiation) Proved by using Lemma~\ref{ss} and Lemma~\ref{ext} in 
 similar manner as for propositional tautology.
 
 
\end{itemize}


\subsection{kTg: Local Deducibility Simulates Global Deducibility}
\textbf{Proof Goal} $Ax \cup KDAxioms \vdash f \Rightarrow \exists p. f \in p \wedge Ax \vdash_l p$


Proof of this direction follows a similar idea from the previous 
direction: we do induction on gtt rules. The tricky 
part of this proof is to come up with the proving goal. 
We defined local deducibility to carry around a list of proved items which 
requires us to store all the proving processes in it for the final 
proved formula. For instance, if we want to use modus ponens to get 
$f2$ from $f1 \rightarrow f2$ and $f1$, under the KGproof notation 
we can only notate $f2$ to be locally deducible by $Ax \vdash_l [f1, f1\rightarrow f2, f2]$.
We have to have $f1 \rightarrow f2$ and $f1$ in the list before the appreance of $f2$, even if 
$f1 \rightarrow f2$ and $f1$ are straight from axioms. 

\begin{itemize}
 \item (Id) Ignored because gtt does not take any assumptions ($\Gamma$) in the desired goal. \item (Ax)
 \item (Ax) 1). Classical propositional logic(CPL) axioms (propositional tautologies) and K axiom are proved by 
 direct substitution and appending; 2). Dual axiom is a bit more complicated because we need to prove 
 two directions separately using KGproof rules then put them together using propositional tautologies and 
 modus ponens. We put the two directions together by propositional tautologies:
 $A\rightarrow (B \rightarrow (A \wedge B))$, where A and B respectively represent 
 one direction of the implication in Dual axiom. We finish the whole proof by substitution of axioms. 
 \item (MP) Proved using KGproof's modus ponens rule directly.
 \item (Nec) Proved using KGproof's necessitation rule directly.
\end{itemize}


\section{Conclusion and Future Work}
In our report, we proved that global deducibility (with no assumption, $\Gamma = \emptyset$) 
and local deducibility are equivalent in HOL4. 
This proof enables further proofs about modal logic such as completenss proof 
to be written in HOL4. 
We also documented our failed attempt with the analysis of the failure, which 
is a good example of showing how to identify logical errors in proofs in modal logic. 

For future work, we can implement the helper lemmas for completeness proof 
and prove the completeness of modal logic using our global and local deducibility 
models. 


\bibliographystyle{eptcs}
\bibliography{generic}
\end{document}
